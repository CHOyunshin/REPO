\documentclass[a4paper]{article}

\usepackage[pages=all, color=black, position={current page.south}, placement=bottom, scale=1, opacity=1, vshift=5mm]{background}
\usepackage[margin=1in]{geometry} % full-width

% AMS Packages
\usepackage{amsmath}
\usepackage{amsthm}
\usepackage{amssymb}
\usepackage{qcircuit}
\usepackage{physics}
% Unicode
\usepackage[utf8]{inputenc}
\usepackage{hyperref}
\hypersetup{
   unicode,
%   colorlinks,
%   breaklinks,
%   urlcolor=cyan, 
%   linkcolor=blue, 
   pdfauthor={HyungGyu Min, Hakbae Lee},
   pdftitle={A simple article template},
   pdfsubject={A simple article template},
   pdfkeywords={article, template, simple},
   pdfproducer={LaTeX},
   pdfcreator={pdflatex}
}

% Vietnamese
%\usepackage{vntex}

% Natbib
\usepackage[sort&compress,numbers,square]{natbib}
\bibliographystyle{naturemag}

% Theorem, Lemma, etc
\theoremstyle{plain}
\newtheorem{theorem}{Theorem}
\newtheorem{corollary}[theorem]{Corollary}
\newtheorem{lemma}[theorem]{Lemma}
\newtheorem{claim}{Claim}[theorem]
\newtheorem{axiom}[theorem]{Axiom}
\newtheorem{conjecture}[theorem]{Conjecture}
\newtheorem{fact}[theorem]{Fact}
\newtheorem{hypothesis}[theorem]{Hypothesis}
\newtheorem{assumption}[theorem]{Assumption}
\newtheorem{proposition}[theorem]{Proposition}
\newtheorem{criterion}[theorem]{Criterion}
\theoremstyle{definition}
\newtheorem{definition}[theorem]{Definition}
\newtheorem{example}[theorem]{Example}
\newtheorem{remark}[theorem]{Remark}
\newtheorem{problem}[theorem]{Problem}
\newtheorem{principle}[theorem]{Principle}

\usepackage{graphicx, color}
\graphicspath{{fig/}}

%\usepackage[linesnumbered,ruled,vlined,commentsnumbered]{algorithm2e} % use algorithm2e for typesetting algorithms
\usepackage{algorithm, algpseudocode} % use algorithm and algorithmicx for typesetting algorithms
\usepackage{mathrsfs} % for \mathscr command

\usepackage{lipsum}

\usepackage{import}

% Author info
\title{Variable Selection for linear model using quantum algorithm}
\author{YeonWoo Seo$^{12}$ \and Hakbae Lee $^1$}

\date{
   $^1$ Yonsei University \\ \texttt{annie9908@yonsei.ac.kr, hblee@yonsei.ac.kr}\\%
   $^2$ Korea Quantum Computing \\ \texttt{hyunggyu.min@kqchub.com}
}

\begin{document}
   \maketitle
   
   \begin{abstract}

   \end{abstract}

   \tableofcontents

%% --> import function으로 chapter별로 .tex를 나누어서 컴파일 
% \import{./TEX}{01}
% \import{./TEX}{02}
% \import{./TEX}{03}
% \import{./TEX}{04}
% \import{./TEX}{05}
\section{Introduction}





%% /clear page, 
\clearpage
\section{Problem: Real Estate Portfolio Optimization Problem}


Quantum computing can optimize $2^n$ combinations through $n$ qubits by giving 1(select) or 0(reject). Therefore, the binary variable selection method can be excellently applied to fields that optimize portfolios, such as stocks. Various quantum-based companies such as IBM, Multiverse, and QCWare are currently stepping up their portfolio optimization projects. Of course, when quantum computers are developed, algorithm trading will be possible, considering countless companies' stock prices and market conditions. However, generating and applying such algorithms at the current level is impossible.

Therefore, the real estate market in Korea is the area where quantum computers can be most efficiently applied to the portfolio optimization sector through NISQ (Noisy Intermediate-Scale Quantum) currently. In fact, according to the paper(?????) , the Korean real estate market shares very similar characteristics to the stock market. Unlike the stock market, where transactions are possible up to the decimal point, real estate is a unique asset that exists only one at a time, so it is suitable to express as a binary form. In addition, budget constraints are currently impossible when optimizing portfolios.  investment advisors and asset managers can raise the amount of capital they want to form a portfolio with real estate, so they can be free from budget constraints.








%% /clear page, 
\clearpage

\section{Portfolio Optimization using QUBO}
\subsection{QUBO and QISKIT Portfolio Optimization} 

%% 여기에 QUBO 간단 설명 필요 

The following is the portfolio optimization problem presented in the Qiskit document, which implements the typical mean-distribution portfolio optimization through QUBO. 

\begin{align*}
\min \ \ \, \quad &qx^{T}\displaystyle\sum x - \mu^{T}x \\ 
\text{subject to}\quad &x\in\{0,1\}^n\,and\,1^Tx=B \quad(2.1)
\end{align*}
 
Here, $x$ is a binary variable indicating whether to select each asset. It has a value of $x[i]=1$ for selection and $x[i]=0$ for non-selection. $\mu$  is a vector representing the expected return on each asset. $\displaystyle\sum$  is the covariance matrix between the returns of each asset. $q$ is someone's risk preference. Finally, $B$  is a budget, which means that up to A assets can be selected. However, this example of Qiskit document is overly simplified compared to the typical portfolio optimization problem in financial markets. The reasons are as follows.
\begin{enumerate}

%이진 표현 방식 
\item Although particular asset is expressed as a binary variable to convert portfolio optimization into a QUBO problem, it is not realistic to include only one or zero assets in the portfolio. Even if investors want to include assets with high returns and low volatility  at a high rate in the portfolio, it is difficult to achieve true optimization result under the unrealistic constraints of including only one or zero asset. 

%비율 계산 방식 
\item In general, for the average-distribution optimization, the proportion of the asset in the entire portfolio is optimized instead of deciding whether a particular asset is included or not. If you think about a portfolio that includes stocks, bonds, foreign exchange, and raw materials, it is more realistic to decide the proportion of bonds in the total portfolio than to decide whether to include one contract or not.

%B개만 선택 가능 
\item The above example specified only $B$ assets can be selected out of total of $n$  possible assets. However, there are various financial products in the real financial market, including derivatives, and institutional investors can access all of these products. Therefore, it is not realistic to limit the number of total assets in advance and to limit the number of assets that can be invested. 

%예산 제약 없음 
\item In the above example, there are no constraints on total available capital. In reality, no matter how good an asset with high returns and low volatility is, it is impossible to invest if the amount per contract exceeds the available capital. However, the example assumes that investors can include one of every asset without regarding the price. 

% 리스크, 수익률 매트릭스 구하는 방법 
\item  The $Q$ matrix of QUBO stands for risk, and the $\mu$ matrix stands for return. IBM QISKIT used a covariance matrix for $Q$ and return rate at a specific time $\mu$. However, a correlation may be insufficient as a measure of risk (?????), and considering returns only at a certain point in time is insufficient for constructing a portfolio that should consider enormous data from past years. Therefore, new proposals for $Q$ and $\mu$ are needed.

\end{enumerate}

% \begin{figure}[H]
%     \centering
%     \includegraphics[width=1.0\linewidth]{figure5.jpg}
%     \caption{Real Datasets Experiments}
%     \label{fig:Figure 5}
% \end{figure}



\end{document}
